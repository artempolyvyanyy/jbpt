\section*{Log Representativeness Measures}
\setcounter{subsection}{0}

Entropia allows users to evaluate an event log with respect to
its generative system based on completeness, coverage,~\cite{Kabierski2023Addressing}
and log representativeness approximation (LRA)~\cite{Karunaratne2024Role}.
This analysis can be done for specific event data aspects: activities, directly-follows relations, or traces.

\subsection{Completeness}

You can calculate the log completeness~\cite{Kabierski2023Addressing} with the following command.
\begin{lstlisting}[style=CL]
>java -jar !jbpt-pm-entropia-1.5.jar! @-l@=log3.xes &-com&
\end{lstlisting}

\textbf{Output Screen:}%chnage
\lstinputlisting[style=DOS]{screens/screen_(-l1).txt}

\subsection{Coverage}

To calculate the log coverage for activities and directly-follows relations only, use the following command.
\begin{lstlisting}[style=CL]
>java -jar !jbpt-pm-entropia-1.5.jar! @-l@=log3.xes &-cov& &-act& &-dfr&
\end{lstlisting}

\textbf{Output Screen:}%chnage
\lstinputlisting[style=DOS]{screens/screen_(-l2).txt}

\subsection{LRA}

To analyze the trace-based log representativeness approximation in silent mode, execute this command.

\begin{lstlisting}[style=CL]
>java -jar !jbpt-pm-entropia-1.5.jar! @-l@=log3.xes &-lra& &-tr& &-s&
\end{lstlisting}

\textbf{Output Screen:}%chnage
\lstinputlisting[style=DOS]{screens/screen_(-l3).txt}


